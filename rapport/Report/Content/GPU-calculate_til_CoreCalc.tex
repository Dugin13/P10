\section{GPU-calculate til CoreCalc}
I dette projekt vil CUDAfy blive brugt, til at øge regnekræften i open source programmet \textit{Funcalc} der kan findes på hjemmesiden \cite{FuncalcHome}.
\textit{Funcalc} en udvidelse til Corecalc, der er en implementering af et regneark funktionalitet lavet i sproget C\#, det er lavet som et forskning prototype som ikke er ment for kunne blive brugt i stedet for de officielse versioner, såsom Microsoft Excel.

Klassen \textit{GPU\_ func} i \textit{GPU\_ calculate} mappen er hvor det meste arbejde ligger fra dette projekt. For at kunne bruge det jeg har fremstillet, er der også tilført noget kode i klassen \textit{Function}.

\subsection{GPU\_ func}
Klassen \textit{GPU\_ func} er der seks funktioner og en konstruktør.

konstruktøren bliver brugt til at hente information om GPU'en der bruges til at bestemme om en blok er nok, hvis ikke hvor mange blokke skal der så bruges. Grunden til at information bliver hentet når man laver klassen er for minimere tiden funktion skal bruge på udregning, da jeg har observeret tager en god potion tid at hente denne information.

\subsection{makeFunc}
\textit{makeFunc} og \textit{makeFuncHelper} funktionerne bruges til at fremstille en opskrift med hvordan hvordan GPU'en skal udregne. Denne liste har \textit{X} antal a fire fire tal som er en enkle udregning (+,-,*,/). Tal første og tredje tal er hvad variabler der skal gøres noget med, det andet tal er for at bestemme hvilken udregning der skal gøres (+,-,*,/) og det fjerde og sidste bliver brugt til at bestemme om resultatet skal lige ligges i en midlertidig variable eller om den skal ligges i resultat listen.

For at give et eksemple kan vi tage regnestykket \textit{(1+2)*(3+4)}, det her ses om en kommando for GPU funktion hvor tallene bliver brugt til at bestemme hvilken kolonne, den skal tag værdien fra. Den kunne komme til at se sådan ud: (1,1,2,-1),(3,1,4,-2),(-1,3,-2,0). 1+2 er blevet lavet om til (1,1,2,-1), 3+4 er blevet om til (3,1,4,-2) og ()*() er lavet til (-1,3,-2,0). Grunden til at der står minus -1 og -2 ved udregningen for 1+2 og 3+4, er at minus værdier bliver brugt til at beskrive midlertidig variabler og 0 for output.

\subsection{findnumberOfTempResult}
Dette er en simple funktion der går gennem en opskrift og finder det antal af midlertidig resultater der skal bruges i opskriften.

\subsection{calculate}
Når man skal bruge en GPU gennem CUDA og CUDAfy skal man gøre forskellige ting, disse ting bliver gjort her. Variablerne \textit{numberOfTempResult}, \textit{SizeOfInput}, \textit{AmountOfNumbers} og \textit{numberOfFunctions} bliver lavet til at starte med. \textit{SizeOfInput} er hvor mange koloner der er med i input, \textit{AmountOfNumbers} er hvor mange rækker der med i input.

Der bliver lavet to array \textit{output} og \textit{tempResult}. \textit{tempResult} Bliver brugt til holde midlertidig resultater for GPU regne stykke, grundet dette bliver gjort her, er at mængden af midlertidig i en funktion kan variere og efter hvad jeg har kunne finde på nettet giver CUDAfy ikke mulighed for at lave et array på run time.

Det første der bliver udregnet er, hvor mange block og tråde der skal bruges, alt efter hvad hardware kan holde til.

Derefter vil array få allokeret plads på GPU, Hvorefter vil de der har nødvendig data blive send over. Derefter vil selve GPU funktion blive kaldt. Derefter vil output data blive hente tilbage fra GPU og til sidst vil den hukommelse der er brugt på array blive frigivet.

\subsection{GPUFunc}
\textit{GPUFunc} er funktion der vil blive kørt på GPU. Den har tre ting der gør for hver punkt i opskriften. Først vil den finde de variabler den skal bruge, midlertidig eller fra input, så vil den gøre noget med de to variabler den har fundet, +,-,*,/, og til sidst vil den finde ud af hvor output skal lægges hen.


\subsection{kode i Corecalc Function klasse}
Koden der findes i \textit{Function} for \textit{Corecalc}, der har bundet GPU koden sammen til resten af \textit{Corecalc}. Der er tre forskellige steder at kode er blevet sat ind, selve \textit{Function} har fået en private \textit{GPU\_func GPU} der bliver initierets når \textit{Function} bliver. For at GPU funktion skal kunne blive kaldt, er den blevet sat ind i tabellen af funktioner.

Den sidste del af koden i denne klasse ligger i \textit{GPUFunction}, koden her er samlelede mellem \textit{Corecalc} og \textit{GPU\_ func}. Koden opgave er tage funktions kaldet input og lave det til information som \textit{GPU\_ func} klassen vil kunne bruge til at udregne med, det gør den ved at hente det beskrevet array, lave det om til et double array, derefter tager den det andet input og laver om til en opskrift. Med opskriften og double array kan den kalde \textit{calculate} der giver et double array tilbage med resultaterne, til sidst vil den fremstille et \textit{Corecalc} \textit{value} array, hvor resultatet vil blive kopieret over i og derefter vil blive sent tilbage.