\section{Intro}
I tidligere Projekt blev der kigget på, om GPU'ens regnekraft kunne bruges i flere programmer for at øge ydeevne. I dette projekt var det også fundet, at det ikke var nemt at programmer til en GPU, fordi metoderne til at kode til en GPU ikke har en godt dokumentation og er kompliceret at lave. Derfor vil jeg i dette projekt prøve at at lave en simple programmering metode for at kunne bruge en GPU's regnekraft GPU, uden man skal havde de store viden inden for det. Regneark bliver brugt af mange og er forholdsvis nemt at programmere, derfor kunne det være interessant, at lave en funktion til et regneark program for at udregne med en GPU. Regneark programment jeg der blev brugt i dette projekt er \textit{Funcalc}\textit{Funcalc} der er open source.
Inden jeg går i gang med at lave GPU funktion vil jeg kigge på tre API for at programmere til en GPU (CUDA, CUDAfy og C++ AMP), for at se hvad der virker godt i forhold til matrix Multiplatikon for at give et godt overblik over hvad der skal bruges når udvidelsen til regnearket skal laves.

Andre har også prøvet at lave det nemmere at kode til en GPU (nogen ref her og eksempler). Det største problem jeg fandt mens jeg arbejde på projektet, var, at igennem CUDA, CUDAfy og C++ AMP ikke gav mulighed for at sende en funktion til GPU, jeg kunne heller ikke finde en smart måde at compile GPU kode på run time, og at når man koder i Excel, bliver det gjort på runtime af programmet der køre excel dokumentet. Måden jeg løste dette problem var ved at sende en opskrift der blev levet efter et abstrakt syntaks træ sammen med data der skal arbejdes på. Med den information vil funktion vide hvordan den skal udregne på den sendte data, dette giver mulighed for at "kode" til en GPU på run time uden at der skal compiles noget på run time. Et anden problem, dog mindre, var at jeg ikke kunne finde løsning på hvordan et program kunne håndtere at bruge forskellige tråde mængder på run time, dette problem jeg jeg kun løst på en meget simple måde der også kan klare forskellige GPU korts.