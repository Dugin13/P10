\section{Intro}
Alt kode kan finde på min GitHub profil Dugin13, her er et link til selve projektet \cite{GitHubProjekt}

I tidligere Projekt \cite{P9} kiggede på forskellen mellem en CPU, der bliver brugt til hverdagen og som er god til sekventielle beregningen, og GPU, der for det meste buges i grafiskes programmer til beregning af pixel. I tidligere Projekt blev det fundet, at det ikke var nemt at programmer til en GPU, fordi metoderne til at kode til en GPU ikke har en godt dokumentation og er kompliceret at lave. Et papir fundet i løbet af det projektet viste også at maget af tiden bliver brugt på at overføre data til GPU \cite{lee2010debunking}. 

Derfor er formålet med dette projekt at lave en simple programmering metode for at kunne bruge en GPU, uden man skal havde de store viden inden for det. For at kunne det vil et regnearks program, der bliver brugt af mange og det er forholdsvis nemt at lave funktioner. Derfor kunne det være interessant, at lave en funktion til et regneark program, der vil ligge udregningen over på GPU. Regneark programment der blev brugt i dette projekt er \textit{Corecalc and Funcalc}. \textit{Corecalc and Funcalc} er et open source regnearks program lavet som et platform for at eksperimenter med teknologi og nye funktioner.

Inden der vil blive lavet på \textit{Corecalc and Funcalc}, vil der blive kigge på tre API for at programmere til en GPU (CUDA, CUDAfy og C++ AMP), for at se hvad der virker godt i forhold til matrix Multiplatikon for at give et godt overblik over hvad der skal bruges når udvidelsen til regnearket skal laves.

Andre har også prøvet at lave det nemmere at kode til en GPU, for at give et eksempel kunne være \textit{Chestnut}\cite{Chestnut2012}. Det første problem blev fundet da der blev arbejde på projektet var, at igennem CUDA, CUDAfy og C++ AMP ikke gav mulighed for at sende et funktion udtryk til GPU. Et andet problem der kom frem var, at når man arbejdere i et regneark program som en bruger, vil man som regel genskrive brugeren en funktion flere gange, eller kunne det være at funktion skal udregnes flere gange på grund af ændringer i arket. Det andet problem kunne blive løst ved at sende en indkodning af et udregnings udtryk der leves efter et abstrakt syntaks træ. Med den information vil funktion vide hvordan den skal udregne på den sendte data, dette giver mulighed for at "kode" til en GPU på run time uden at der skal compiles noget på runtime. Der er dog nogen der har kigget på ideen med at compile GPU kode på runtime, såsom \textit{Firepile} \cite{Firepile2011} der er en udvidelse til \textit{Scala}

Et anden problem, dog mindre, var det ikke rigtig finde løsning på, hvordan et program kunne håndtere forskellige mængder af tråde på run time til en GPU, dette problem bliver løst på en meget simple måde der også kan klare forskellige GPU korts.