\section{Intro}
Alt kode kan finde på denne GitHub profil Dugin13, her er et link til selve projektet \cite{GitHubProjekt}.

I tidligere Projekt \cite{P9} blev der undersøgt på forskellen mellem en CPU, som bruges til hverdagen og er god til sekventielle beregningen og GPU, der for det meste bruges i grafiskes programmer til beregning af pixel. I tidligere projekt blev det fundet, at det ikke var nemt at programmer til en GPU, fordi metoderne til at kode til en GPU ikke har en god dokumentation og er kompliceret at lave. En artikel fundet i løbet af det projektet viste også, at meget af tiden bliver brugt på at overføre data til GPU \cite{lee2010debunking}. 

Ud fra det tidligere projekt vil formålet med dette projekt, være at lave en simple programmering metode for at kunne bruge en GPU, uden man skal havde de store viden inden for dette, for at kunne det vil et regnearks program, der bliver brugt af mange og det er forholdsvis nemt at lave funktioner. Derfor kunne det være interessant, at lave en funktion til et regneark program, der vil flytte udregningen over på GPU. Regneark programment der bruges i dette projekt er \textit{Corecalc and Funcalc}. \textit{Corecalc and Funcalc} er et \textit{open source} regnearks program lavet som et platform for at eksperimenter med teknologi og nye funktioner.

Før GPU funktion til \textit{Corecalc and Funcalc} vil blive lavet, vil en test blive udarbejdet, der kigges på tre API for at programmere til en GPU (\textit{CUDA}, \textit{CUDAfy} og \textit{C++ AMP}), for at se hvad der virker godt i forhold til matrix Multiplatikon og give et godt overblik over, hvad der skal bruges, når udvidelsen til regnearket skal laves.

Andre har også prøvet at lave det nemmere at kode til en GPU, for at give et eksempel kunne være \textit{Chestnut}\cite{Chestnut2012}. Det første problem blev fundet da der blev arbejde på projektet var, at igennem \textit{CUDA}, \textit{CUDAfy} og \textit{C++ AMP} ikke gav mulighed for at sende et funktion udtryk til GPU. Et andet problem der kom frem var, at når man arbejdere i et regneark program som en bruger, vil man som regel omskrive en funktion flere gange, eller kunne det være at funktion skal udregnes flere gange på grund af ændringer i arket. Det andet problem kunne blive løst ved at sende en indkodning af et udregnings udtryk der laves efter et abstrakt syntaks træ. Med den information vil funktion vide, hvordan den skal udregne på den sendte data, dette giver mulighed for at "kode" til en GPU på \textit{run time} uden at der skal \textit{compiles} noget på \textit{run time}. Der er dog nogen der har kigget på ideen med at \textit{compile} GPU kode på \textit{run time}, såsom \textit{Firepile} \cite{Firepile2011} der er en udvidelse til \textit{Scala}.
Et andet problem var at man ikke kunne finde en rigtig løsning på, hvordan et program håndtere forskellige mængder af tråde på \textit{run time} til en GPU, problemet bliver løst på en meget simple måde, der også kan klare forskellige GPU \textit{cards}.

I kapitel \textit{Relateret Arbejde} \ref{relateret_arbejde} har arbejdet inden for det samme område som dette projekt vil.

I kapitel \textit{Matrix Multiplatikon} \ref{MM} vil en beskrivelse af koden, som bliver brugt til at teste hvilken GPU bibliotek der virker bedst sammen med det regneark programmet \textit{Corecalc and Funcalc}. Ved at fremstille et program der kan gøre en matrix multiplatikon for hvert bibliotek, der bliver kigget på.

I kapitel \textit{Test af Matrix Multiplatikon} \ref{Test_MM} kan udstyret på computeren, som er blevet testet på dokumenteres, samt resultaterne fra testene med matrix multiplatikon af de forskellige biblioteker.

\textit{GPU Funktion} \ref{GPU_F} giver en beskrivelse af hvordan funktion, der vil blive fremstillet til \textit{Corecalc and Funcalc} vil virke i programmet for brugeren.

\textit{GPU-calculate til CoreCalc} \ref{GPU_CC} beskriver klassen, der er lavet for at kunne udregne GPU i \textit{Corecalc and Funcalc} ved at sende en indkodning til GPU af at udtryk, som GPU ville kunne bruge til at udregne, det ønskede resultat ud fra data det bliver sendt med.

For at finde ud af om funktion, der er fremstillet i dette projekt kan bruges i stedet for hvordan man normal vil udregne i et regneark, er en test blevet lavet af den, som kan ses i kapitel \textit{Test af GPU-calculate} \ref{TEST_GPU}.

For at opsummere hvad der er blevet fundet i dette projekt, og kigget på om jeg har svaret på, om man kan bruge et regnearks program til at gøre det nemmere at kode til en GPU, er der en \textit{Diskussion} \ref{DIS}. kapitel \ref{KON} er \textit{Konklusion}.

\textit{Fremtidige Arbejde} \ref{FA} kigger på, hvad der er af muligheder for at videre udvikle på funktion lavet til \textit{Corecalc and Funcalc} og nogen teste ville være interessant at foretage.