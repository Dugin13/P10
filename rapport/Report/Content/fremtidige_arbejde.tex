\section{Fremtidige Arbejde}
\label{FA}
Første ting kunne være at lave lidt om på GPU funktion, så den kunne tage værdier der er den samme for alle udregninger, på dette tidspunkt kan GPU funktion kun arbejde med et \textit{array} der består af input. Men hvis GPU funktion gav lov til at sende et ekstra \textit{array} med globale værdier alle tråde skal bruge, kunne det hjælpe på overførsel tid og spare på pladsen på GPU. Dette kunne gøres ved at tillade, at når man giver lov til at markeret en celle ved bruge af GPU funktion, der sådan bliver til den global værdi, måden de globale værdier kunne pointet til i indkodning, ville være at bruge de positive tal der ikke bliver bugt for at beskrive input data. Eksempelvis kunne man skrive en GPU funktion sådan: $=GPU(A1:B10, (1+2)*C1)$. her vil den lave et \textit{array} over \textit{A1:B10} hvor hver tråd har den data i og sende et \textit{array} med \textit{C1} som alle tråde vil tilgå, for at få fadt på denne værdi.

GPU funktinon kunne blive videreudvikle, så den kan håndterer \textit{if()} og \textit{rand()}. Måden den kunne klare problemet med \textit{if()}, er ved at tilføre ekstra værdier ind ind i indkodningen. Eksempelvis kunne man se \textit{if()} som en split i en udregning, vil den nye værdi indkodning holde styr på hvilken gren man befinder sig på. Derudover skal der også lavet så funktion på GPU, som er i stand til at udregne et boolean udtryk. Udvidelsen for at læse boolean udtryk burde ikke være det store problem, men udvidelsen for at håndtere \textit{if()} er muligvis ikke liget til, siden GPU funktion skal kunne håndtere at hoppe i indkodning. \textit{CUDAFY} ser ikke ud til at have en tilfældigheds funktion i sig, så derved kan \textit{rand()} ikke gøres på GPU siden, man kunne lave nogen tilfældige tal og sende med til GPU hvis den skal et tilfældigt tal, dette ville dog sænke hurtigheden for GPU, siden at testene viser at, der helst skal være flere udregninger en mængde af data man sender, ville det ikke være hurtigere at gøre udregningerne på GPU. Dog ser det ud til at \textit{CUDA} har  en form for tilfældigheds generator med navn \textit{cuRAND}, denne funktion kunne måske blive til rådighed for \textit{CUDAfy} i fremtiden. 

Et forslag ville være at GPU funktion kan tage en \textit{funcalc} funktion, skrevet i et funktion ark, i sted for at kun kan tage et \textit{CoreCalc} udtryk. Ved at gøre dette ville bruger venligheden muligvis blive bedre. Dette burde kunne lade sig gøre, da et abstrakt syntaks træ bliver lavet i funktion arket i \textit{funcalc}.

En anden interessant ide kunne være, at give brugeren mulighed for at direkte bruge cellerne som punkt for, hvad der skal ligges sammen, GPU(a1:c5 , a1+b1+c1). Så skal GPU funktion selv finde ud af om de celler i udtrykket inden for input \textit{array} eller om det er noget udenfor der, hvilket vil gøre det til en global værdi. 

Unikke GPU funktioner kunne blive lavet over kendte algoritme problemer, en af dem kunne være det samme om der er blevet brug til at teste de forskellige sprog: matrix Multiplatikon, koden er næsten lavet til fulde, der mangler nogen små ændringer, såsom at tage matrixer, der ikke har den samme længe og højde, som input kunne denne funktion tag 2 forskellige markeret celler området.

Ideen ved dette projekt var at gøre det nemmere at kode til en GPU, dette blev aldrig testet. Derfor kunne det være interessant at lave en brugertest af GPU funktion for at se, hvad en bruger mener om funktion og om der muligvis kunne laves nogen forbedring på funktion, som ikke er kommet op under projektet arbejdes tid.

Noget andet der også kunne være interessant at teste, ville være om måden at at GPU funktion kan på sin vis \textit{compile} simpel udtryk på run time, ved brug af en indkodning og læser, ville kunne klare sig mod projekter der compiler GPU kode på run time. Dette kunne være interessant at kigge på, for at se om der er hurtigere at lave GPU kode på run time eller om der hurtigere at lave en indkodning som en GPU kan udregne efter.