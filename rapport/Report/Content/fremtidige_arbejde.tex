\section{Fremtidige Arbejde}
På grund af tids præs eller kommet op med ideen forsendt, er der planer om at lave videre på dette projekt.

Først ting kunne være at lave ligt om på GPU funktion så den kunne tage værdier der er den samme for alle udregninger, på dette tidspunkt kan GPU funktion kun arbejde med et array der består af input. Men hvis GPU funktion gav lov til at sende et ekstra array med globale værdier alle tråde skal bruge, kunne det hjælpe på overførsel tid og spare på pladsen på GPU. Dette kunne gøres ved at tillade at når man skriver funktion giver lod til at markeret et celle, der sådan bliver til den global værdi, måden de globale værdier kunne pointet til i indkodning, ville være at bruge de positive tal der ikke bliver bugt for at beskrive input data.

Man kunne også gøre at GPU kan tage en \textit{funcalc} funktion, skrevet i et funktion ark, i sted for at kun kan tage en \textit{CoreCalc} udtryk. Ved at gøre dette ville bruger venligheden muligvis blive bedre. Det burde være relativ simple at gøre da en \textit{funcalc} funktion vist også skulle lave et abstrakt syntaks træ over hvordan funktions skal udregnes, derved kan man løbe dette abstrakt syntaks træ igennem på samme måde det bliver gjort for et \textit{CoreCalc} udtryk. Det kunne også være interessant at give brugeren mulighed for at direkte bruge cellerne som punkt for hvad der skal ligges sammen, GPU(a1:c5 , a1+b1+c1), så skal funktion selv finde ud af om de tal i udtrykket pejler på noget inde i input array eller om det er noget udenfor der, hvilket vil gøre det til en global værdi. 

Der kunne laves nogen bestemt lavet GPU funktioner over kendte algoritme problemer, en af dem kunne være det samme om der er blevet brug til at test de forskellige sprog matrix Multiplatikon, koden er jo næsten lavet til fulde, nogen små ændringer, såsom at tage matrixer der ikke har den samme længe og højde, som input kunne denne funktion tag 2 forskellige markeret celler området.

Ideen ved dette projekt var at gøre det nemmere at kode til en GPU, dette bliver aldrig testet. Derfor kunne det være intersant at lave en bruger til på GPU funktion, for at se hvad en bruger mener om funktion og om der muligvis kunne laves nogen forbedring på funktion som ikke er kommet op under projektet arbejdes tid.

Noget andet der også kunne ver intersant at teste, ville være om måden at et udtryk bliver indkodet, som så en GPU funktion kan læse i dette projekt på run time. Kan stille noget op mod projekter der compiler GPU kode på run time. Dette kunne være interessant at kigge på, for at se om der er hurtigere at lave GPU kode på run time eller om der hurtigere at lave en indkodning som en GPU kan udregne efter.