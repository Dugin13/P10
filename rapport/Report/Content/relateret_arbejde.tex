\section{relateret arbejde}
Det tidligere projekt \cite{P9} blev der kigget på hvad forskellen mellem CPU og GPU og forskellige metoder til at kode til en GPU. Metoden der blev brugt til at teste det med, var ved at fremstille den samme funktion som et benchmark i de forskelle metoder at udregne på, Funktion der blev brugt var \textit{k-means clustering}. Det blev fundet at der er mange måder at gøre GPU programmering, men for at få noget ud af det skal men havde faring og viden først. i prjektet blev der kigget på CUDA, AMP, F\# Brahma OpenCL og Harlan. 

Bogen \textit{Spreadsheet Implementation Technology: Basics and Extensions}\cite{Spreadsheet_Implementation_Technology} af Peter Sestoft er en samling af information inden for programmering af regneark, for at hjælpe programmører der skal i gang med at udvikle til regne ark.

Artiklen \textit{THE GPU COMPUTING ERA}\cite{nickolls2010gpu} giver et indblik over hvordan NVIDIA GPU'er har udviklers sig, samt hvordan man kan kode til dem gennem årende.

Artiklen \textit{A Survey of CPU-GPU Heterogeneous Computing Techniques}\cite{mittalsurvey} kigger på hvad der er sket inden for den videnskabelige ramme inden for Heterogene Computing Teknikker, såsom partitonering af arbejdsbyrde for at udnytte CPU'er og GPU'er til at forbedre ydeevnen og/eller energieffektivitet. Der bliver også kigge på benchmark der kan bruges til at evaluere Heterogene computersystemer.

For at gøre noget ved den dårlige beskrivelse af sprog der er ved GPU programmering har artiklen\textit{GPU Concurrency: Weak Behaviours and Programming Assumptions}\cite{alglave2015gpu} kigget på dem. Disse beskrivelsers af sprog føre til at mange programmører bliver nød til at bruge antagelser, når der laves software til GPU. Har dette papir gennemført en undersøgelse af GPU'er, Ved at bruge \textit{litmus} tests. Der bliver kigget på antagelser i programmering guider og leverandør dokumentation om de garantier, som hardware giver.

Det er ikke kun et problem at kode til en GPU, man skal også kunne vide hvordan information skal gemmes for at få det meste ud af en GPU, som bliver beskrevet i artiklen Artiklen \textit{Adaptive Input-aware Compilation for Graphics Engines}\cite{samadi2012adaptive}.

Til at hjælpe med at fremstille resultater der kunne bruges, er Papiret \textit{Microbenchmarks in Java and C\#}\cite{Microbenchmarks} blevet brugt til at hjælpe med at fremstille test programmer, der kunne give resultater angånde den brugte til på at løbe en funktion igennem.

Web Blog \textit{Collaboration and Open Source at AMD: LibreOffice} \cite{AMD_LibreOffice} beskriver om et projekt der har gjort det samme som dette projekt har planer om at prøve at fuldføre. Her har de brugt \textit{LibreOffice} som regneark programmet til udvikle på og der bliver brugt \textit{OpenCL} for programmering af GPU.



