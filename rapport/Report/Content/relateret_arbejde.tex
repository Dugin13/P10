\section{Relateret Arbejde}
\label{relateret_arbejde}
Det tidligere projekt \cite{P9} blev der kigget på hvad forskellen mellem CPU og GPU, de forskellige metoder til at kode til en GPU. Metoden der blev brugt til at teste det med, var ved at fremstille den samme funktion som et \textit{benchmark} i de forskelle metoder at udregne på, Funktion der blev brugt var \textit{k-means clustering}. Resultatet blev at der er mange måder at programmerer til en GPU, men for at få noget ud af det, skal men havde erfaring og viden ingen for dette område. I projektet blev der kigget på \textit{CUDA}, \textit{AMP}, \textit{F\# Brahma}, \textit{OpenCL} og \textit{Harlan}. 

Artiklen \textit{THE GPU COMPUTING ERA}\cite{nickolls2010gpu} giver et indblik over hvordan \textit{NVIDIA} GPU'er har udviklers sig, samt hvordan man kan kode til dem gennem årende.

Artiklen \textit{A Survey of CPU-GPU Heterogeneous Computing Techniques}\cite{mittalsurvey} beskriver, hvad der er forgået inden for \textit{Heterogene Computing} Teknikker i den videnskabelige ramme. Noget af det der er omtalt er partitonering af arbejdsbyrde for at udnytte CPU'er og GPU'er til at forbedre ydeevnen og/eller energieffektivitet. Der bliver også beskrevet \textit{benchmark} der kan bruges til at evaluere Heterogene computersystemer.

For at gøre noget ved den dårlige beskrivelse af sprogene, der er ved GPU programmering har artiklen \textit{GPU Concurrency: Weak Behaviours and Programming Assumptions}\cite{alglave2015gpu} undersøgt dem. Disse beskrivelsers af sprogene føre til at mange programmører bliver nød til at bruge antagelser ved programmering af software, hvor der gøres brug af GPU. Har denne artikel gennemført en undersøgelse af flere GPU, ved at bruge \textit{litmus} tests. Beskrivelser i programmerings guider og de garantier som leverandørens dokumentation om hardware giver, vil blive undersøgt i omtalte artikel.

Det er ikke kun et problem at kode til en GPU, man skal også kunne vide hvordan information skal gemmes for at få det meste ud af en GPU, som bliver beskrevet i artiklen Artiklen \textit{Adaptive Input-aware Compilation for Graphics Engines}\cite{samadi2012adaptive}.

Til at hjælpe med at fremstille resultater der kunne bruges igennem test af udregnings tid, er Papiret \textit{Microbenchmarks in Java and C\#}\cite{Microbenchmarks} blevet brugt til at hjælpe med at fremstille test programmer, der kunne give resultater angående den brugte tid på at løbe en funktion igennem.

Bogen \textit{Spreadsheet Implementation Technology: Basics and Extensions}\cite{Spreadsheet_Implementation_Technology} er en samling af information inden for programmering af regneark, for at hjælpe programmører der skal i gang med at udvikle til regne ark.

Web bloggen \textit{Collaboration and Open Source at AMD: LibreOffice} \cite{AMD_LibreOffice} beskriver et sammenarbejde mellem dem der har lavet \textit{LibreOffice} og firmaet \textit{AMD}, hvor formålet var at tillade at overføre udregninger i regneark programmet \textit{LibreOffice} til GPU ved at bruge OpenCL. Denne kode blev fremstillet af \textit{AMP} ansatte.

\textit{Spreadsheet optimisation on GPUs} \cite{GPUP2010} er et bachelor projekt som kigger på om det er muligt at bruge GPU til at udregne parallelt i et regnearks program. Programmet som der vil blive udvide på i dette projekt er også \textit{CoreCalc}.

Som dette projekt har planer om at prøve at fuldføre. Her har de brugt \textit{LibreOffice} som regneark programmet til udvikle på og der bliver brugt \textit{OpenCL} for programmering af GPU.